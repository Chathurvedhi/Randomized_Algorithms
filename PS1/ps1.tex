\documentclass[addpoints,12pt]{exam}
\usepackage{amsmath,amsthm,amssymb}
\usepackage{libertine}
\usepackage[libertine]{newtxmath}
\useosf
\linespread{1.05}
\usepackage[T1]{fontenc}  
\usepackage{hyperref}
\usepackage[top=0.5in,bottom=1in,left=1in,right=1in]{geometry}
\usepackage[linesnumbered,ruled]{algorithm2e}
\usepackage[dvipsnames]{xcolor}
\usepackage{tikz}
\usepackage[framemethod=TikZ]{mdframed}
\printanswers
\marksnotpoints

\pointformat{\textcolor{Maroon}{(\thepoints)}}
\qformat{\textcolor{Maroon}{\textbf{Problem~\thequestion}} \hfill
  \textcolor{Maroon}{\textbf{\totalpoints~\points}}}


\newcommand{\course}{{\large CS6170: \textsc{Randomized} \textsc{Algorithms}}}
\newcommand{\pset}[1]{{\large \textsc{Problem} \textsc{Set} \##1}}
\newcommand{\due}[1]{{\textsc{Due}: #1}}
\newcommand{\name}[1]{\textsc{Name}: \textcolor{Blue}{#1}}
\newcommand{\rno}[1]{\textsc{Roll} \textsc{No}: \textcolor{Blue}{#1}}
\pagestyle{plain}

%%%%%%%%%%%%%%%%%%%%%% Your macros come here%%%%%%%%%%%%%%%%%%%%%%%%%%%%%%%%
\newcommand{\E}{\mathbb{E}}
\newcommand{\vect}[1]{\mathbf{#1}}
\newcommand{\F}{\mathbb{F}}
%%%%%%%%%%%%%%%%%%%%%%%%%%%%%%%%%%%%%%%%%%%%%%%%%%%%%%%%%%%%%%%%%%%%%%%%%%%%


\begin{document}
	
	\hrule height 2pt
	\vspace{2mm}
	{\centering \course \par}
	\vspace{1mm}
	{\centering \pset{1}\par}
	\vspace{1mm}
	\noindent \name{Your name} \hfill \textsc{Marks}: \numpoints\\
	\rno{Your roll number}\hfill \due{August 27, 23:59}
	\vspace{2mm}
	\hrule height 2pt
	\vspace{2mm}
	
	\begin{questions}

          \question[3] Suppose that you want to generate a random permutation of
          the sequence of numbers $1$ to $n$. You have at your disposal, a
          source of unbiased random bits. Give an efficient algorithm
          to generate a random permutation using as few random bits as possible
          from the source.
          \begin{solution}
            Type your solution here.
          \end{solution}

          \question[4] While discussing Frievald's algorithm for verifying
          matrix multiplication, the following algorithms were proposed in
          class.
          \begin{enumerate}
          \item \textbf{Algorithm 1}: Given $A$, $B$, $C$, choose a number $j$
            uniformly at random from $\{1,2,\ldots,n\}$ and multiply $A$ with
            the $j^{th}$ column of $B$ and check if it matches the $j^{th}$
            column of $C$ entrywise.
          \item \textbf{Algorithm 2}: Given $A$, $B$, $C$, choose two numbers
            $i$ and $j$ uniformly at random from $\{1,2,\ldots,n\}$ and multiply
            the $i^{th}$ row of $A$ and the $j^{th}$ column of $B$ and check if
            the product is equal to the $(i,j)^{th}$ entry of $C$.
          \end{enumerate}
          Analyze the two algorithms and explain which one is better. Are these
          algorithms better than Frievald's algorithm? What is the running time
          of this algorithm if I want to make the error probability $\epsilon$?
          \begin{solution}
            Type your solution here.
          \end{solution}
		
          \question[5]  An $s$-$t$-cut in a graph is a set of edges such that
          their removal gives a new graph which does not contain a path from $s$
          to $t$. Consider the following modification of Karger's algorithm to
          compute the smallest $s$-$t$-cut in the graph: Choose a random edge in
          the graph such that it not between supernodes containing $s$ and $t$,
          and contract it; keep continuing until the only two supernodes are the
          ones containing $s$ and $t$, and output this as the minimum
          $s$-$t$-cut.

          Show that there are graphs such that the success probability of this
          algorithm finding the minimum $s$-$t$-cut is exponentially small.
          \begin{solution}
            Type your solution here.
          \end{solution}
          
          \question Consider the following randomized algorithm.
          
          \begin{algorithm}[H]
            \SetKwFor{Repeat}{repeat}{times}{}

            Set $X \gets 0$

            \Repeat{$n$}{
              Set $X \gets X+1$ with probability $1/2^X$
            }

            Set $Y \gets 2^X - 1$
            \caption{}
          \end{algorithm}

          \begin{parts}
            \part[5] Compute $\E[Y]$.
             \begin{solution}
               Type your solution here.
             \end{solution}
         
             \part[3] Give a tight bound on the number of bits required to
             represent $X$. Notice that the number of bits required to represent
             $X$ is a random variable.
             \begin{solution}
               Type your solution here.
             \end{solution}
           \end{parts}

           \question[5] A collection of $n$ bits $X_1, X_2, \ldots, X_n$ are
           said to be $k$-wise independent if for every subset $S$ of $k$ bits
           among the $n$, and for $b_1, b_2, \ldots, b_k \in \{0,1\}$, we have
           \begin{align*}
             \Pr\left[\bigcap_{i\in S} X_i = b_i\right] = \prod_{i\in S} \Pr\left[X_i = b_i \right]
           \end{align*}

           Consider the following construction: Let
           $\vect{x}_1, \vect{x}_2, \ldots, \vect{x}_n \in \{0,1\}^\ell$ be $n$
           vectors such that every set of $k$ vectors are linearly independent
           over $\F_2$. Let $\vect{y}\in \{0,1\}^\ell$ be chosen uniformly at
           random. Define $X_i$s as follows:
           \begin{align*}
             X_i = \left( \sum_{j=1}^\ell \vect{x}_{i,j}\vect{y}_j \right) \text{ mod $2$}.
           \end{align*}
           Show that $X_i$s are $k$-wise independent.
           \begin{solution}
               Type your solution here.
             \end{solution}
	\end{questions}
\end{document}